\documentclass[a4paper,12pt]{article}
\usepackage{fullpage}
\usepackage[british]{babel}
\usepackage[T1]{fontenc}
\usepackage{amsmath}
\usepackage{amssymb}
\usepackage{amsthm} \newtheorem{theorem}{Theorem}
\usepackage{color}
\usepackage{float}
\usepackage{listings}
\usepackage{subfig}
\lstset{% parameters for all code listings
	language=Python,
	frame=single,
	basicstyle=\small,  % nothing smaller than \footnotesize, please
	tabsize=2,
	numbers=left,
	framexleftmargin=2em,  % extend frame to include line numbers
	%xrightmargin=2em,  % extra space to fit 79 characters
	breaklines=true,
	breakatwhitespace=true,
	prebreak={/},
	captionpos=b,
	columns=fullflexible,
	escapeinside={\#*}{\^^M}
}
\usepackage{fancyvrb}
\DefineVerbatimEnvironment{code}{Verbatim}{fontsize=\small}
\DefineVerbatimEnvironment{example}{Verbatim}{fontsize=\small}

\usepackage{tikz} \usetikzlibrary{trees}
\usepackage{hyperref}  % should always be the last package

% useful colours (use sparingly!):
\newcommand{\blue}[1]{{\color{blue}#1}}
\newcommand{\green}[1]{{\color{green}#1}}
\newcommand{\red}[1]{{\color{red}#1}}

% useful wrappers for algorithmic/Python notation:
\newcommand{\length}[1]{\text{len}(#1)}
\newcommand{\twodots}{\mathinner{\ldotp\ldotp}}  % taken from clrscode3e.sty
\newcommand{\Oh}[1]{\mathcal{O}\left(#1\right)}

% useful (wrappers for) math symbols:
\newcommand{\Cardinality}[1]{\left\lvert#1\right\rvert}
%\newcommand{\Cardinality}[1]{\##1}
\newcommand{\Ceiling}[1]{\left\lceil#1\right\rceil}
\newcommand{\Floor}[1]{\left\lfloor#1\right\rfloor}
\newcommand{\Iff}{\Leftrightarrow}
\newcommand{\Implies}{\Rightarrow}
\newcommand{\Intersect}{\cap}
\newcommand{\Sequence}[1]{\left[#1\right]}
\newcommand{\Set}[1]{\left\{#1\right\}}
\newcommand{\SetComp}[2]{\Set{#1\SuchThat#2}}
\newcommand{\SuchThat}{\mid}
\newcommand{\Tuple}[1]{\langle#1\rangle}
\newcommand{\Union}{\cup}
\usetikzlibrary{positioning,shapes,shadows,arrows}


\title{\textbf{Separation of Voronoi Areas}}

\author{Jonathan Sharyari, Martin, Sascha}  % replace by your name(s)

%\date{Month Day, Year}
\date{\today}

\begin{document}

\maketitle

\section{Abstract - do we need one?}
I've layouted this from the outlines of a similar report, which required an abstract. I didn't see the point then, and I don't see the point now since we arn't writing a scientific report. But you guys probably know better than me, how this is usually done.


\section{Introduction}
What is a voronoi diagram?
\begin{itemize}
\item
	picture
\end{itemize}
What is a triangulation?
\begin{itemize}
\item
	picture
\item
	The relation to the voronoi diagram (duality)
\item
	another picture
\end{itemize}

What is an arrangement?
\begin{itemize}
\item
	picture
\end{itemize}

Problem description?

\begin{itemize}
\item
	Should it be here? 
\end{itemize}
\section{Background}
The problem is NP-complete

Bounds for solution
\begin{itemize}
\item
	Picture showing the case where |blue| = |red| - 1
\item
	The problem is thought to always be solvable with |blue| = |red|, but this has not been shown. (Or am I wrong?)
\item
	Briefly mention upper bounds, referring to the paper we read. Not really relevant.
\end{itemize}

\section{Our solution}
\begin{itemize}
\item
A graphical gui for input, with input of red dots either by manual placement or by reading a file
\item
Output by drawing delaunay circles and voronoi-edges.
\item
Calculating the arrangement, finding a point in every face and make gurobi solve the least number of points needed to cover each circle.
If this didn't solve the problem, add more circles. We should mention how these new circles are chose (I'm not sure if I remember that part correctly even). We should also say something on why we believe this method will be able to find an optimal/near-optimal solution (I guess "Mr Rote said so" isn't really good enough. A paper maybe?)
\end{itemize}
\section{Tools}
\begin{itemize}
\item
Git
\item
Qt
\item
Gurobi
\item
C++
\item
CGAL
\item
CMAKE
\end{itemize}

\section{Abstract program layout}
How are the different parts of the program put together. Right now, we basically have a CGAL-part, a QT-part and a gurobi-part. Then we have one class, Geometry, that puts them all together. This is actually quite good.

We also have the file myGraphics.cpp, which uses CGAL directly. We should probably avoid this, and use functions in Geometry.cpp instead.

A picture showing file dependencies would be nice.

\section{Problems encountered}
\begin{itemize}
\item
Getting all the different parts (tools) to work and compile together. Especially cmake. Gurobi could also be mentioned, as it needs a C++ wrapper or whatever you call it.
\item
Finding a point inside a face. If we end up using the algorithm we have now, that algorithm would need a quite nice picture. And pictures are nice.
\item
Hopefully nothing more, but I don't think we will be that lucky :)
\end{itemize}
\section{Results}
What did we manage to do. What didn't we manage to do. This is a little bit difficult to outline at this stage.

\section{Result}
\begin{itemize}
\item
A picture of the gui
\item
Some problems, and the resulting solution.
\item
How well does it perform?
\end{itemize}

\section{Discussion}
Even more difficult at this stage


\section{references}
Keeping the references I used in the last project, since they probably follow a good convention.

[1] H. G. Cobb, J. J. Grefenstette (1993)

Genetic Algorithms for Tracking Changing Environments

Proceedings of the 5th International Conference on Genetic Algorithms

[2] A. Simões, E. Costa (2002)

Using Genetic Algorithms to Deal with Dynamic Environments: A Comparative Study of Several Approaches Based on Promoting Diversity

GECCO '02 Proceedings of the Genetic and Evolutionary Computation Conference

[3] S. Yang (2007)

Genetic Algorithms with Elitism-Based Immigrants for Changing Optimization Problems

Lecture Notes In Computer Science, 2007, volume 4448, pages 627-636
\end{document}

